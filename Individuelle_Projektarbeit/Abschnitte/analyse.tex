\section{Analyse \& Bewertung}
In diesem Kapitel soll die Arbeit analysiert und bewertet werden. Somit ist dies mit dem Projekt-Output gleichzusetzen. Hierfür gibt es mehrere methodische und inhaltliche Ansätze. Zum einen wird der Kunde zu einem Review der geleisteten Arbeit eingeladen. Hier ist es besonders wichtig die objektive Reaktion zu betrachten und weitere Verbesserungsvorschläge, Wünsche und Anregungen aufzunehmen. Mit der Fachabteilung werden die Kennzahlen des digitalen Geschäftsprozesses aufgenommen. Ein Vergleich dieser mit den Daten aus Abschnitt \ref{KennzahlenIstProzess} identifiziert den messbaren Mehrwert der Applikation. Eine anschließende Bewertung schließt das Projekt ab.

\subsection{Review des Prototyps}
Der sogenannte Review einer Entwicklungsphase ist aus Projektsicht sehr wichtig. Der Kunde, also in diesem Fall die Fachabteilung der Abrechnung, hat das fachliche Expertenwissen und kann einschätzen, wie gut die entwickelte Applikation tatsächlich ist. Der Review beginnt mit einem Rückblick auf den eingegrenzten Kernprozess in Abschnitt \ref{Kernprozess}. Die dazugehörigen von der Fachabteilung während der Kennzahlenaufnahme des Istprozesses erhobenen Teilschritte werden erneut dargestellt. Anschließend beginnt die Live Demonstration des entwickelten Prototypen. Jeder anwesende Teilnehmer wurde gebeten, während der Vorstellung alle Fragen und Anregungen direkt zur Sprache zu bringen. 
\subsubsection{Reaktion der Fachabteilung}
Die erste Reaktion der Fachabteilung ist äußerst positiv. Die Aufmachung der Applikation und die dazugehörigen entwickelten Funktionen sehen die Anwesenden als sehr nützlich an. Es wird direkt darauf eingegangen, welche manuellen Teilschritte durch diesen Prototypen nicht mehr erforderlich sind (siehe Abschnitt \ref{KennzahlenSollProzess}). Einer der ersten Fragen der Fachabteilung ist, ob man die Applikation ab Februar direkt in den produktiven Einsatz nehmen könnte. Hier wird durch die entsprechenden IT-Verantwortlichen gegengesteuert und diese Möglichkeit ausgeschlossen. Dies hängt mit dem Softwarestatus des Prototypen zusammen und vielen Anforderungen, die für das erste Release notwendig und noch nicht entwickelt sind. Hierzu gehören als Beispiel die \glqq nice to have\grqq-{} und \glqq out of scope\grqq- Anforderungen. Besonders die Punkte Dokumentenmanagement (vgl. Abschnitt \ref{EcoDMS}) und der in Abschnitt \ref{Übersichten} beschriebene fehlende Mehrwert für den Berater sind hier die Hauptkriterien.
\subsubsection{Neue Anforderungen \& Wünsche}
Die Demonstration der Software hat bei den Beteiligten zusätzlich einige neue Ideen ausgelöst. Zum einen wurden viele der in dem Prototyp unbeachteten Anforderungen in Abbildung \ref{scope} erneut als sinnvoll angefragt. Die Funktionsweise der Dokumentenarchivierung beispielsweise war ein zentrales Thema. Hier wird im Detail gewünscht, dass viele Informationen wie das Eingangsdatum, die in StaffIT hinterlegte Kostenstelle und die Prüfunterschrift automatisch mit dem Dokument verknüpft werden. Ein weiterer Kernpunkt war die Budgetprüfung, die in dem Prototyp keine Beachtung gefunden hat. Anhand der in \glqq StaffIT pro\grqq{} gepflegten Informationen lässt sich diese Prüfung zusätzlich voll automatisieren. Der letzte große Punkt auf der Wunschliste war das Thema Übersichten und Reporting aus Abschnitt \ref{Übersichten}. Da sich die Fachabteilung hier nur schwer etwas vorstellen konnte, wurde ein Beispiel eines anderen Geschäftsprozesses herangezogen und das Dashboard erläutert. Dies wurde als sehr positiv wahrgenommen. 
\subsection{Kennzahlen des digitalisierten Prozesses}\label{KennzahlenSollProzess}
Die Kennzahlen des Prototypen werden anhand mehrerer Fallbeispiele vorgenommen. So wird mithilfe von Testdaten der Prozess mehrfach simuliert und die nötigen Prozessschritte durchgespielt. Die erste Erkenntnis ist die nötige Anzahl an Teilprozessschritten. Diese ist nun bei den nachfolgenden neun manuell nötigen Teilschritten.
\begin{enumerate}
	\item Rechnung einmal ausdrucken
	\item Leistungsnachweis zweimal mal ausdrucken 
	\item Angaben (Dokumente) auf Fehler, Unvollständigkeit, richtige Angaben kontrollieren
	\item Berater in der Vertragsübersicht herausfiltern
	\item Bestellung auswählen
	\item auf Übersichtsblatt Tage/Stunden eintragen (Excel)
	\item Budget/Laufzeit kontrollieren (Excel)
	\item auf monatlichem Blatt Tage/Stunden und das Rechnungsdatum eintragen (Excel)
	\item Bestellung in \grqq StaffIT pro\grqq{} öffnen
\end{enumerate}
Die Simulation wird zehn mal mit unterschiedlichen Beratern und Projekten durchgeführt. Als Ergebnis ist eine durchschnittliche Durchlaufzeit von einer Minute und acht Sekunden festzuhalten. Es wurde insbesondere darauf geachtet ähnliche Testfälle wie die der Kennzahlenaufnahme des Ist-Prozesses in Abschnitt \ref{KennzahlenIstProzess} zu kreieren. Die Schwankungen der einzelnen Durchläufe lag bei acht Sekunden. Das ist auf die unterschiedlichen realistischen Testfälle zurückzuführen. 

\subsection{Optimierungsgrad der Applikation}
Für die Bestimmung des Optimierungsgrades werden die Kennzahlen der beiden Prozess-Versionen gegenübergestellt. Zusätzlich wird eine Hochrechnung des zeitlichen Mehrwertes für die Datenerhebung eines Monates vorgenommen. 
\subsubsection{Gegenüberstellung der Kennzahlen}
Die einzelnen manuellen Prozessschritte konnten durch die in Abschnitt \ref{Prozessentwicklung} beschriebene Möglichkeit des Eliminierens und des Automatisierens von vorher sechzehn auf nur noch neun Teilschritte reduziert werden. Somit entfallen durch die Digitalisierung hier insgesamt sieben Prozessschritte. Auch die durchschnittliche Durchlaufzeit hat sich von vier Minuten und dreizehn Sekunden auf eine Minute und acht Sekunden reduziert. Dies macht einen zeitlichen Vorteil von drei Minuten und fünf Sekunden. Dies ist umgerechnet eine prozentuale Verbesserung von rund 73,85 Prozent. \\\\
Die Hochrechnung basiert auf aktuellen Daten der COM Software GmbH im Dezember 2017. Diese hat mit entsprechenden Beratern 223 offene und zu bearbeitende Verträge zu verzeichnen. Bei dem aktuellen nicht digitalisierten Prozess macht das insgesamt eine Bearbeitungsdauer von rund 921 Minuten, also 15,35 Stunden für die reine Datenaufnahme der Berater aus. Mithilfe des digitalisierten Datenaufnahme-Prozesses werden nur noch rund 240 Minuten benötigt, also rund vier Stunden. Somit ist hier eine monatliche zeitliche Ersparnis von bis zu 11,35 Stunden möglich. Bei diesen Zahlen ist weiterführend nur der aktuelle Funktionsstand des Prototypen beachtet. Sollten weitere Funktionalitäten implementiert werden, kann der Nutzen weiter steigen. 
\subsubsection{Einordnung Reifegrad}
Auch in Bezug auf den Reifegrad kann dieser Teilprozess eine klare Steigerung verzeichnen. Der Istprozess wurde während der Aufnahme begründet in Stufe eins des Reifegrad Modells in Abbildung \ref{fig:bpcc} eingestuft. Der Prototyp befindet sich nun auf dem Level drei des Modells. Der Prozess ist demnach definiert und kann so direkt eingesetzt werden. Ob Anpassungen an besondere Anforderungen nötig sind muss eine entsprechende Beta-Phase zeigen. Für das vierte Level des Modells fehlt der aktuellen Applikation noch die automatische Prozessmessung. Hier lässt sich ein komplettes Monitoring im IBM Business Process Manager implementieren. Der sogenannte Business Monitor erstellt daraufhin auf einen Prozess abgestimmte Auswertungen anhand weitere Optimierungsmöglichkeiten erkannt werden können.


\subsection{Bewertung}
Final betrachtet ist das Ergebnis des Prototypen sehr positiv zu bewerten. Die für den Prototyp ausgewählten Anforderungen sind erfolgreich umgesetzt und entsprechen den Vorstellungen der Fachseite. Somit ist ein sehr gutes Bild der Problemstellung entstanden und der Optimierungsansatz für alle Beteiligten greifbar. Besonders unterstützt wird dieser Projekterfolg durch die Reduktion der Durchlaufzeit um 73,85 Prozent und die damit verbundene Eliminierung und Automatisierung von sieben manuellen Prozessschritten. Auch die Einordnung der Applikation in das Reifegradmodell auf Level drei ist als positiv einzustufen. Hier ist ganz klar darauf hinzuweisen, dass damit der Anspruch eines Prototypen vollends erfüllt ist. Besonders erfolgreich ist zusätzlich die Zusammenarbeit zwischen Fachabteilung und Entwicklung hervorzuheben. Die Gespräche waren stets positiv und mit einem enormen Mehrwert für den weiteren Projektverlauf verbunden.
