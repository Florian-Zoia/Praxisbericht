\section{Wie werden Leads generiert ?}
Jedes Unternehmen muss für sich selbst entscheiden, ab wann ein Kunde ein Lead ist oder nur ein Interessent. Manche Firmen sehen einen Interessenten erst als Lead an, wenn ein Formular ausgefüllt und die Personenbezogenen Daten gesammelt wurden. Für andere ist es bereits ein Lead wenn sich aktiv für das Unternehmen interessiert wird. Nachdem dieses Kriterium Unternehmensintern geklärt wurde können Unternehmen beginnen Lead Generierung in ihre Website, Social Media und Werbung zu integrieren. 

\subsection{Website}
Für die Generierung von Leads ist die eigene Website eine der wichtigsten Schwerpunkte, da Unternehmen hier die Freiheit haben alles selbst zu gestalten. Die anzuwendenden Methoden wie Formulare, CTA's, verschiedene Downloads, Landing-Pages und Bewertungen sowie die Selbstdarstellung können frei gewählt werden. Genauso kann entschieden werden wie stark sich auf die einzelnen Methoden konzentriert wird.

\subsubsection{Formulare}
Zu einem Lead gehören Kontaktdaten sowie eine Möglichkeit die Person zu kontaktieren. Die am meisten genutzte Methode, um an diese Daten zu kommen, sind die Formulare. 
\newline 
Ein großer Teil des Inbound Marketings besteht darin dem Interessenten oder der Interessentin immer nur Bruchstücke der gewünschten Informationen zu liefern. Sobald jemand echtes oder großes Interesse zeigt sind Personen bereit diese Daten im Austausch gegen mehr Informationen heraus zu geben. Um an weitere Informationen zu gelangen kann ein Download bereitgestellt oder es muss sich für einen Newsletter angemeldet werden. 
\newline
An diesem Punkt spielen Formulare eine essenzielle Rolle, da so weder ein Kundenkonto erstellt werden muss, noch ist es notwendig, dass sich eine Person bei einer Website registriert. Somit wird es dem potenziellen Lead sehr leicht gemacht Daten anzugeben. Es muss kein aufwendiger Anmeldeprozess durchlaufen werden und die Informationen stehen nach nur drei bis vier Angaben zu eigenen Person bereit. Dabei können die wichtigsten Informationen, wie Name, Vorname, E-Mail Adresse und eventuell auch die aktuelle Stelle im Betrieb abgefragt werden. Formulare können auch mehr abfragen, wobei nicht notwendiger Weise alle Felder Pflichtfelder sein müssen, außer wenn es für das Unternehmen notwendig ist diese Informationen zu erhalten. Zum Beispiel bei einer Hintergrund Überprüfung, wo mehrere Informationen, zur Person, essentiell sind. Jedoch kann es Interessenten abschrecken, wenn sie zu viele Daten über sich heraus geben müssen. Genauso ist es nicht für jedes Unternehmen notwendig so viele Daten über den potenziellen Lead zu erhalten. 
\newline
Ist das Formular ausgefüllt können Kunden gleich auf die gewünschten Informationen zugreifen und das Unternehmen bekommt alle notwendigen Daten, mit welchen der potenzielle Lead erreicht werden kann.

\subsubsection{Bereitstellung von Downloads}
Um Kunden oder potenzielle Leads Neugierig und interessiert an der eigenen Firma zu halten, ist es wichtig nicht alle nötigen Informationen zu sich selbst auf einmal heraus zu geben. 
\newline 
Befindet sich beispielsweise ein Artikel auf der eigenen Website und die Benutzer, welche sich den Artikel durchgelesen haben, möchten mehr Informationen zu dem Thema erhalten, so kann ein Dokument bereitgestellt werden, auf welchem weiter ins Detail gegangen wird. Dabei sollte der Artikel so aufgebaut sein, dass die weitergehenden Informationen beworben werden. Somit bekommt der/die Leser/in einen Einblick in die Materie. Allerdings werden tiefer gehende Informationen vorenthalten, welche erst beim Download des Dokuments oder eines ähnlichen Mediums zur Verfügung stehen.
\newline 
Ist der/die Leser/in interessiert und würde sich gerne weitreichender Informieren, so ist er/sie gezwungen sich die weitergehenden Informationen zu Downloaden. Allerdings sollte an dieser Stelle ein Formular eingebaut werden, welches die Kontaktdaten abfragt. Somit erhalten sowohl die Kunden ihre gewünschten Informationen und können sich anhand dieser bei dem Unternehmen melden, um einen Auftrag abzuschließen. Gleichzeitig hat das Unternehmen nun die benötigten Informationen um den Lead, welcher mit dem ausfüllen den Formulars entstanden ist, weiter zu pflegen, sollte sich dieser nicht für einen Auftrag melden. Sei dies mit persönlich zugeschnittenen E-Mails, basierend auf dem wofür sich der Lead interessiert hat oder welchem bestehenden Kunden der Lead am ähnlichsten ist. Dadurch kann sich das Unternehmen viel spezifischer selbst bewerben. Wenn zu diesem Zeitpunkt die Probleme des Leads feststellbar sind, so kann sich das Unternehmen auch darauf konzentrieren und gleich mit Problemlösungen kommen, um attraktiver für den Lead zu wirken.

\subsubsection{CTA - Call to Action}
Ein CTA, zu deutsch ein Aufruf zum Handeln, ist gut platzierte Werbung die dafür sorgt, dass die einzelnen Besucher der eigenen Website auch auf dieser bleiben. Der Kunde wird aufgefordert weitere Artikel zu lesen oder sich etwas herunterzuladen. Dabei gibt es die verschiedensten Varianten von CTA's. 
\newline
Zum einen gibt es die so genannten Slide-In CTA's, welche wie ein Pop-Up Fenster funktionieren. Sie erscheinen allerdings erst, wenn ein gewisser Punkt auf der Seite erreicht wird. Das kann bedeuten, dass sobald ein Artikel fertig gelesen wurde ein Fenster erscheint, was auf einen weiteren Artikel hinweist, welcher dem Besucher der Website gefallen könnte. Zum anderen kann dies auch eine direkte Anfrage sein, dass sich der Kunde bei der Firma melden soll, wenn Interesse an dem Thema besteht. 
\newline
Für die COM Software GmbH kann dies zum Beispiel bedeuten, dass die Besucher der Homepage, nachdem sie sich eines der Geschäftsfelder durchgelesen haben, aufgefordert werden die Xing Seite des Vertriebsleiters zu besuchen. Dort gibt es weitere Informationen für die Besucher und somit informiert sich die Person immer weiter über das Unternehmen. Ist dies geschehen könnten auf verschiedenste Arten Werbung geschaltet werden, sodass die Person das Unternehmen nicht vergisst und die COM Software für die Planung im nächsten Projekt mit einbezieht. 
\newline 
Das Ziel bei dieser Vorgehensweise ist es immer nur Bruchstücke der Informationen, die der eventuelle Neukunde haben möchte, preiszugeben. Somit informiert sich dieser immer mehr, wenn Interesse besteht und die COM Software kann sich auf diese Person konzentrieren. Dabei kann es sein, dass es für den Kunden von Nöten ist sich für einen Newsletter oder Verteiler anzumelden, um weitere Informationen zu erhalten oder auf diese zugreifen zu können. Ist der Interessent erst einmal im System kann mit Mails oder Nachrichten für die eigene Firma geworben werden. Dabei ist es wichtig eine Kundenbeziehung aufzubauen. Dies kann zum einen dadurch geschehen, dass die Mails aus dem Newsletter oder Verteiler personalisiert sind und von einer echten Person verschickt werden. Dies erzeugt ein Gefühl der Betreuung, da auf die Mails oder Nachrichten geantwortet werden kann. Bei automatisch generierten Mails und Newslettern wissen die Kunden, dass sie in einem Datenverzeichnis mit anderen Unternehmen hinterlegt wurden. Dadurch kommt ihnen die Werbung unpersönlich vor und es kann passieren, dass die Mail oder Nachricht weder gelesen noch geöffnet wird. 

\subsection{Landing-Page}
Um Neukunden oder Interessenten Informationen zu liefern und sie als Lead zu gewinnen kann es sehr hilfreich sein eine oder mehrere Landing-Pages in die eigene Homepage zu integrieren. Das Ziel einer Landing-Page besteht darin informativ zu sein. Meist erreicht man diese Seiten nur über Werbung oder Links und kann nicht von der eigentlichen Homepage zu diesen navigieren. 
\newline 
Der Inhalt einer Landing-Page ist sehr spezifisch auf eine Zielgruppe ausgelegt. Dadurch, dass diese Seite nur extern erreichbar ist bekommen auch nur bestimmte Zielpersonen oder Unternehmen einen Zugriff. Der Aufbau und Inhalt einer Landing-Page soll das Interesse dieser einen Zielgruppe wecken, sodass sich die Unternehmen entweder mehr für die ((((eigene Firma)))) interessieren oder gleich kontaktieren, um in die Verkaufsgespräche zu kommen. Dabei werden Informationen zur (((eigenen Firma))) und Dienstleistungen herausgegeben, allerdings nur stückweise, um im Gegenzug für Informationen die Kontaktdaten der jeweiligen Besucher zu erhalten.
\newline 
Somit können mehrere Landing-Pages in die eigene Homepage integriert werden, um die verschiedenen Zielgruppen und Angebote der COM Software GmbH abzudecken. 

\subsubsection{Blog Artikel}
Um mehr Besucher auf die eigene Website zu bekommen ist es oftmals hilfreich einen Blog zu starten. Dieser ist in der Regel sehr zeitintensiv, allerdings 

\subsubsection{Reviews und Selbstpromotion}

\subsection{Social Media}

\subsection{Search Engine Optimation}

\subsection{Kickback und Follow-Up Emails}



\section{RPA - Robotic Process Automation im Inbound Marketing}
Um Leads zu generieren müssen die richtigen Personen angeschrieben werden. Es reicht nicht die eigenen Angebote an die Firmenmail eines Unternehmens zu schicken. Dort gehen die Informationen meist verloren, werden nicht von den richtigen Leuten gesehen oder ignoriert. Um dieses Problem zu umgehen versuchen Marketing Teams sich gleich an die bestimmenden Personen zu wenden, um Ihnen die eigenen Angebote zu präsentieren. Allerdings ist es nicht immer einfach die richtigen Mail Adressen zu haben, insbesondere wenn das Unternehmen noch kein Kunde war. Des weiteren ist dies eine zeitintensive und beschwerliche Aufgabe. 
\newline 
Damit diese Zeit gespart werden kann und sich das Marketing und Vertriebs Team auf wichtigere Themen konzentrieren kann, die das eigene Unternehmen voran bringen, gibt es so genannte RPA's. Diese sind Programme oder Bots, die die Mail Adressen und Kontaktdaten zu Personen finden, welche in das Kunden Schema der jeweiligen Firma passt. Somit kann gleich die richtige Person kontaktiert werden. Sollte diese Person die eigene Firma nicht kennen, so ist es möglich, dass die eigene Firma bei Neukunden präsent gemacht wird. Wenn das Marketing Team mit Hilfe dieser Informationen eine Kundenspezifische Mail verschickt oder anderweitig Kontakt aufnimmt, könnte es sein, dass sich das angesprochene Unternehmen zu der eigenen Firma informiert und sogar als Neukunde gewonnen werden kann. 
\newline 
Hierbei ist es wichtig die bestehenden Kunden gut zu dokumentieren und die Daten ordentlich zu pflegen. Mit Hilfe dieser Daten kann ermittelt werden, ob es sich überhaupt lohnt bei einem Unternehmen anzufragen. 	